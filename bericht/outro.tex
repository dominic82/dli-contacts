\section{Beurteilung des Projektes}
\label{sec:beurteilung}

Zusammenfassend lässt sich das Projektergebnis als Erfolg verzeichnen. Zum einen natürlich, weil das erzeugte jABC-Modell den Anforderungen der Aufgabenstellung genügt. Der Großteil des Erfolges ist jedoch sicherlich der Erfahrung geschuldet, welche während des Projektes gesammelt werden konnte. So zeigten sich während der Bearbeitung die Besonderheiten einer Datenmigration in Zusammenhang mit der Verwendung von Webservices. 

So muss zum einen darauf geachtet werden, ob gewählte Webservices überhaupt die Methoden zu Verfügung stellen, welche für die Aufgabenstellung benötigt werden. Während es in diesem Fall zum Beispiel möglich war bei SAP nach einzelnen Attributen von Personen zu suchen, bot der Webservice von Google diese Möglichkeit nicht. Hier mussten bei jeder Anfrage alle Kontakte zurück"=geliefert werden. Ein Aussortieren fand erst im Anschluss lokal statt.

Zum anderen ist darauf zu achten, dass im Zielsystem auch alle Attribute des Quellsystems gespeichert werden können. Unbedingt muss in diesem Zusammenhang auch darauf geachtet werden, eine ausreichende Zuordnung der Attribute des Quell- und Zielsystems zu spezifizieren. In einigen Fällen kann es daher vorkommen, dass sich bestimmte Informationen bei einer Migration nicht übertragen lassen. Im speziellen Fall dieses Projektes gab es hinsichtlich dieses Aspektes keine Probleme.

Nicht zuletzt kann das Projekt auch hinsichtlich der Zusammenarbeit der Projektteilnehmer als Erfolg verzeichnet werden. 