%Schriftgr��e, Layout, Papierformat, Art des Dokumentes
\documentclass[12pt,oneside,a4paper]{scrartcl}
%Das Paket erzeugt ein anklickbares Verzeichnis in der PDF-Datei.
\usepackage{hyperref}

%Einstellungen der Seitenr�nder
\usepackage[left=3cm,right=4cm,top=2.5cm,bottom=2.5cm,includeheadfoot]{geometry}

%Zeilenabstand
%\renewcommand{\baselinestretch}{1.25}\normalsize %f�r 1,5 zeilig
\usepackage{setspace}

%Einr�ckung eines neuen Absatzes
\parindent 0pt
\parskip 6pt

%neue Rechtschreibung
\usepackage[ngerman]{babel}

%Umlaute erm�glichen
\usepackage[utf8]{inputenc}

%Kopf- und Fu�zeile
\usepackage{fancyhdr}
\pagestyle{fancy}
\fancyhf{}

%Linie oben
\renewcommand{\headrulewidth}{0.5pt}

%Fu�zeile rechts bzw. au�en
\fancyfoot[C]{\thepage}
%Linie unten
\renewcommand{\footrulewidth}{0.5pt}

%Bilder einfuegen
\usepackage{graphicx}

%Bilter etc. positionieren
%\usepackage{float}
\usepackage{placeins}

%Code
\usepackage{listings}
%fuer fette typewriter Schrift
\renewcommand{\ttdefault}{pcr}

%Farben (benoetigt fuer listings)
\usepackage{color}
\usepackage[usenames,dvipsnames]{xcolor}

%%%%%%%%%
% Style %
%%%%%%%%%
%
% Hier ein paar Befehle und Konventionen um das Schriftbild einheitlich zu halten
%
% Bitte folgende standard Befehle nutzen:
%
% \emph{} zum Hervorheben von Text
%
% \url{} um Web-Adressen darzustellen 
%
% \label{xxx} um einen referenzierbaren Eintrag zu erstellen und \ref{xxx} um auf
% einen referenzierbaren Eintrag zu verweisen
% WICHTIG: in einer Umgebung wie "figure" immer erst \caption setzen, DANACH \label
%
% \cite{} um auf einen Bibliothek-Eintrag zu verweisen (s. literatur.tex) 
%
% Codeformatierung
\newcommand{\codeformat}{\small\ttfamily}
\newcommand{\jkeycolor}{\color{DarkOrchid}\bfseries}
%
% vor jeder lstlisting-Umgebung aufrufen, 1. caption, 2. label
\newcommand{\javalstset}[2]{
\lstset{% general command to set parameter(s)
	basicstyle=\codeformat, % print whole listing small
	keywordstyle=\jkeycolor,
	% underlined bold black keywords
	identifierstyle=, % nothing happens
	commentstyle=\color{Gray}, % white comments
	stringstyle=\color{Blue}, % typewriter type for strings
	showstringspaces=false % no special string spaces 
	tabsize=3,
% ACHTUNG: lstlisting stellt Tabulatorzeichen mit dar! %
	language=Java,
	captionpos=b,
	caption={#1},
	label={#2},
	frame=trbl,
	breaklines=true,
	breakatwhitespace=true
	}
}
% WICHTIG: listings mit \begin{lstlisting}[float] code... \end{lstlisting} aufrufen
%
% ein einzelnes Wort in Codeformatierung darstellen
%
% \lstinline{code} !!!Sinnvollster Befehl, verfehlt allerdings seine Wirkung, wenn man ihn kapselt
%
\newcommand{\code}[1]{\codeformat{#1}}% für Klassen und normalen Code
\newcommand{\codejkey}[1]{{\jkeycolor\codeformat{#1}}}% für java keywords
%
% Anführungszeichen
\newcommand{\gaensefuesse}[1]{\glqq{#1}\grqq}
%
% wsimport
\newcommand{\wsimport}{\emph{wsimport}}
%%%%%%%%%%
% Style/ %
%%%%%%%%%%

\begin{document}
%Beginn der Titelseite
\begin{titlepage}
\begin{small}
\vfill {Technische Universit\"at Dortmund\\ 
Fakult\"at Informatik \\
Lehrstuhl 5, Prof. Dr. Bernhard Steffen \\
Wintersemester 2012/13}
\end{small}


\begin{center}
\begin{Large}
\vfill {\textsf{\textbf{
Aktuelle Themen der Dienstleistungsinformatik\\\ \\
Projekt: Kontakte
}}}
\end{Large}
\end{center}

\begin{small}
\vfill Markus Marzotko, Thorben Seeland, Dominic Wirkner\\ 
\today
\end{small}

\end{titlepage}
%Ende der Titelseite
  
%Inhaltsverzeichnis (aktualisiert sich erst nach dem zweiten Setzen)
\tableofcontents
\thispagestyle{empty}
%Beginn einer neuen Seite
\clearpage
%Anderthalbzeiliger Zeilenabstand ab hier
\onehalfspacing

\subsection{Das Projekt}
\section{Einleitung}

- Im Rahmen der Vorlesung DLI ... Dieser Bericht erläutert im Detail ...

- In diesem Kapitel
	- Zunächst Aufgabe darstellen
	- Daraufhin Skizzierung der Projektstruktur und erste allgemeine Überlegungen

- In Kapitel 2 Darstellung der einzelnen Projektbausteine im Detail: 
	- Beginnt mit den Connectoren: SAP-Connector, Google-Connector
	- Dann Bereich der SIB-Programmierung: im Allgemeinen und im Speziellen bezogen auf das Projekt
	
- In Kapitel 3 Darstellung des fertigen jABC-Modells
	- Erklärung Modellaufbau
	- Erklärung von Design-Entscheidungen (Meta-SIBs), Error-Branches
	
- Abschließend in Kapitel 4 eine kurze Zusammenfassung sowie Beurteilung über den Erfolg des Projektes	


\subsection{Projektbeschreibung}
- fiktives Szenario: Umstellung der informationstechnischen Infrastruktur bei einem Unternehmen
	- In diesem Projekt der Teil, welcher die Migration von Stammdaten der Kontakte des Unternehmens betrifft
		- Kontakte gliedern sich in drei Gruppen: Kunde, Lieferant und Angestellter
	- Migration erfolgt von SAP nach Google. Dabei kommt nicht-triviale Strategie zum Einsatz (Karteilleichen):
		- Kontakte nur nach Bedarf übertragen
		- Erst Google prüfen, dann versuchte Migration von SAP, sonst ganz neuen Datensatz bei Google anlegen

\subsection{Erste Überlegungen}
- Als erstes Zergliederung des Projektes, Aufgabenverteilung, Berührungspunkte, eingesetzte Technologien

\subsubsection{Projektstruktur}
- Analyse der Projektstruktur / Identifikation von autonomen Mechanismen:
	- zur Aufgabenverteilung an einzelne Mitglieder: Spezialisten schaffen, Verantwortungsbereiche festlegen
	- Identifikation von Berührungspunkten der Teilbereiche
	- Aufteilung des Projektes (siehe Grafik)
	
\subsubsection{Das Kontakt-Objekt als Schnittstelle}
- Das Kontakt Objekt:
	- Bildung einer repräsentativen Schnittmenge von typischen Kontaktattributen
		- Wahl der Attribute
		- ID-Speicherung für Möglichkeit der Anwendung von komplexeren Migrations-Strategien
	- Modellierung einer allgemeinen Klasse Kontakt als Schnittstellenobjekt
		- zwischen den Funktionen, innerhalb des jABC-Modells
		
\subsubsection{Eingesetzte Technologien}
- Verteilte Programmierung -> VCS git
- verwendete Sprache Java -> JUnit-Tests zur Verifikation der Schnittstelle zw. SIBs und Konnektoren
- Apache-Maven im Hinblick auf Unterstützung von build und deploy Aufgaben
	- autom. Tests im build
	- unterstützung beim deploy durch erzeugung eines jar-Archivs inkl. notwendiger Abhängigkeiten
		- eingentlich Bad Practice
		- Im Umgang mit jABC simpler


\section{Unsere Projektbausteine}
 - Im Folgenden Beschreibung, besondere Überlegungen und Probleme(!)/Entscheidungen der einzelnen Elemente





\subsection{SAP-Connector}
\subsubsection{Aufgabe des SAP Connectors}

Die Aufgabe des SAP Connectors besteht darin, die Datenbank im ES Workplace nach einem 
gegebenen Filter zu durchsuchen. Die gefilterten Datensätze werden zurückgegeben. 

\subsubsection{Aufbau und Programmablauf}

Alle Operationen dienen dazu, Datensätze aus den Datenbanken des ES Workplace auszulesen. Der ES Workplace stellt hierfür 
eine Datenbank mit Testdatensätzen bereit, sowie die erforderlichen Webservices, um eine Verbindung zur Datenbank, 
die Authentifizierung und die Suche selbst durchzuführen. Die erforderlichen Webservices liegen als XML Dateien vor, werden 
jedoch zur Verwendung in SIBs als Java Klassen benötigt. Dafür wird Java API for XML Web Services verwendet, welches seit
Version 1.6  Bestandteil der Java SE ist. Mittels Jax-WS können in der Konsole durch einen wsimport Befehl alle benötigten 
Webservices von XML Dateien in Java Klassen umgewandelt werden. 
\ \\
Der SAP Connektor erhält ein Objekt vom Typ Contact. Dieses Objekt enthält die vom Benutzer geforderten Filtereinstellungen 
betreffend Vorname, Nachname, Postleitzahl, Stadt, Straße und Firma. Außerdem wird der Typ nach Kunde, Lieferant und 
Mitarbeiter unterschieden. Dieses Objekt wird verwendet, um in der Datenbank des ES Workplace, nach dem entsprechenden Typ,
Datensätze zu filtern. In Abhängigkeit von der Typbestimmung werden unterschiedliche Webservices ausgeführt:
\ \\

\textbf{Lieferant}
\begin{itemize}
\item Find Supplier by Name and Address
\item Read Supplier Basic Data 
\end{itemize}

\textbf{Mitarbeiter}
\begin{itemize}
\item Find Employee by Elements
\item Find Employee Address by Employee
\end{itemize}

\textbf{Kunde}
\begin{itemize}
\item Find Customer by Elements
\end{itemize}

\ \\
Zunächst soll hier das Programm am Beispiel des Kundenwebservices beschrieben werden. Anschließend werden die Besonderheiten
bei den Webservices für Mitarbeiter und Lieferant erläutert.

\ \\

Zunächst müssen die notwendigen Objekte der Klassen des Webservices ``Find Customer by Elements'' erstellt werden. Mittels
einer Klasse die immer mit dem Schlüsselwort SERVICE beginnt, kann ein Objekt erzeugt werden auf dem eine getBinding Methode
ausgeführt wird. Das erstellte Verbindungsobjekt wird dann zu einem Objekt vom Typ Binding Provider gecastet, auf welchem dann
mittels getRequestContext() Methoden Passwort und Username gesetzt werden können. Für die Authentifizierung ist normalerweise
auch noch eine Webadresse anzugeben, allerdings ist hier die Adresse der Testdatenbank bereits enthalten. Das 
Verbindungsobjekt verfügt über eine einzige Methode, die ein Objekt mit Suchkriterien übergeben bekommt und eine Liste 
von Suchergebnissen zurückliefert. Die Suchkriterien können mit einfachen bereits integrierten set Methoden gesetzt werden,
die Suchergebnisse über get Methoden ausgelesen werden. Anschließend werden die Suchergebnisse als Liste von Kontaktobjekten
zurückgegeben.





\subsubsection{Probleme}

\subsubsection{Ausblick}

\subsection{Google-Connector}
\javalstset{}{}
\FloatBarrier
\subsubsection{Die Bibliothek \emph{gdata}}
Die Google-Bibliothek \emph{gdata} ist eine frei verf\"ugbare Bibliothek zum erstellen von
 Clientapplications für die Services der Google-Cloud.
\emph{gdata} kapselt die Webservices komplett in Java-Klassen, so dass ein importieren
 (z.\ B.\ mit \wsimport) nicht mehr notwendig ist.

In der Beschreibung der \emph{gdata}-Bibliothek wird beschrieben, wie man aus der zip-Datei
 ein JAR compilieren kann, da dies bei mir in mehreren Versuchen nicht geklappt hat,
 haben wir die pragmatische Lösung gewählt und die in der zip-Datei enthaltenen JARs einzeln
 in das Projekt eingefügt\cite{GO02}.
\FloatBarrier
\subsubsection{Authentifizieren und Verbinden mit \emph{gdata}}
Google bietet zwei Authentifizierungsverfahren an
\begin{enumerate}
	\item\emph{OAuth}
	\item Username und Passwort
\end{enumerate}
\emph{OAuth} ist ein Service, der bei erfolgreicher Anmeldung ein Token erstellt, mit dem
 der Client von Google bereitgestellte Services aufrufen und sich authentifizieren kann.
So muss der Client die Anmelde-Daten des Nutzers nicht speichern, sondern nur den Token.
In Abbildung \ref{fig:google_oauth} wird ein Beispiel f\"ur die Nutzung von \emph{OAuth}
 dargestellt.
\begin{figure}[h!]
\includegraphics[width=\textwidth]{Bilder/googleOauth.jpg}
\caption{Nutzungsbeispiel f\"ur \emph{OAuth}\cite{GO01}}
\label{fig:google_oauth}
\end{figure}

Da wir unseren eigenen Account nutzen und die Daten nicht sicherheitskritisch sind, haben
 wir die zweite Variante gew\"ahlt und authentifizieren uns bei jedem Service-Aufruf mit
 Username und Passwort.

Die Authentifizierung wird für ein \lstinline{ContacsService}-Objekt, wie in Listing
 \ref{lst:authpwbeispiel} abgebildet, einmal durchgeführt, danach wird sie vom Framework
 automatisch durchgeführt.
 \\
In der Abbildung wird ein \lstinline{ContactsService}-Objekt erstellt.
Der \lstinline{String}-Parameter \lstinline{servicename} dient hier als Identifikator für den Nutzer.
\\
Nachdem das \lstinline{ContactsService}-Objekt erstellt ist, wird der Service mit Nutzername und
 Passwort angemeldet.
 
\javalstset{Beispiel für die Authentifizierung ohne \emph{OAuth}}{lst:authpwbeispiel}
\begin{lstlisting}[float=h!t]
ContactsService myService;
myService = new ContactsService(servicename);
try {
	myService.setUserCredentials(username, password);
} catch (AuthenticationException e) {
	e.printStackTrace();
}
\end{lstlisting}

\FloatBarrier
\subsubsection{Kontakte suchen}
Die \emph{gdata}-Bibliothek bietet die Möglichkeit, Kontakte wie in Listing \ref{lst:searchQuery}
 mit Angabe eines \lstinline{Query}-Objekts herunterzuladen.
Im Codebeispiel wird das \lstinline{Query}-Objekt \lstinline{myQuery} für die hinter der URL \lstinline{feedURL}
 liegenden Kontakte initialisiert, danach wird es durch die \lstinline{"group"}-Option angewiesen nur
 Kontakte aus der Gruppe mit dem \lstinline{String}-Identifikator \lstinline{groupId} herunterzuladen.
Nachdem \lstinline{myQuery} konfiguriert ist, wird mit dem authentifizierten \lstinline{ContactsService}
 \lstinline{myService} der Service-Aufruf per \lstinline{query()}-Befehl ausgeführt.
Die Klasse \lstinline{Query} kann allerdings nur zwischen Gruppen unterscheiden, jedoch nicht nach
 anderen Kriterien wie z.\ B.\ dem Vornamen oder dem Nachnamen eines Kontakts filtern.

\javalstset{Kontaktsuche per \lstinline{Query}}{lst:searchQuery}
\begin{lstlisting}[float=h!t]
URL feedUrl = new URL(contactsURL);
Query myQuery = new Query(feedUrl);
ContactFeed resultFeed = null;
// Gruppe
String groupId = null;
// Parameter Contact filter
switch (filter.getType()) {
case CUSTOMER:
	groupId = customerGroupURL;
	break;
case SUPPLIER:
	groupId = supplierGroupURL;
	break;
case EMPLOYEE:
	groupId = employeeGroupURL;
	break;
default:
	break;
}
myQuery.setStringCustomParameter("group", groupId);
// submit request
resultFeed = myService.query(myQuery, ContactFeed.class);
\end{lstlisting}

Das Suchen von Kontakten geschieht in unserem Projekt durch das Herunterladen aller Kontakte
 einer Gruppe und anschließendem Aussortieren \gaensefuesse{von Hand}.

\FloatBarrier
\subsubsection{Kontakte einf\"ugen}
Das Einfügen von Kontakten ist über ein erstelltes Service-Objekt 

\javalstset{Kontakt-Objekt (\lstinline{ContactEntry}) erstellen}{lst:createContact}
\begin{lstlisting}[float=h!t]
// Create the entry to insert
ContactEntry contact = new ContactEntry();
contact.setTitle(new PlainTextConstruct(contactInfo.getFirstname()
		+ contactInfo.getLastname()));
\end{lstlisting}

\javalstset{Namen in ein Kontakt-Objekt (\lstinline{ContactEntry}) einfügen}{lst:ccsetname}
\begin{lstlisting}[float=h!t]
// Name
Name name = new Name();
name.setFamilyName(new FamilyName(contactInfo.getLastname(), null));
name.setGivenName(new GivenName(contactInfo.getFirstname(), null));
contact.setName(name);
\end{lstlisting}

\javalstset{Benutzerdefinierte Einträge zu einem Kontakt-Objekt (\lstinline{ContactEntry}) hinzufügen}{lst:cccustomEntry}
\begin{lstlisting}[float=h!t]
// Firma
if (contactInfo.getCompany() != null) {
	ExtendedProperty company = new ExtendedProperty();
	company.setName(DLI_GoogleContactsConnector.company);
	company.setValue(contactInfo.getCompany());
	contact.addExtendedProperty(company);
}
\end{lstlisting}

\javalstset{Den Kontakt einer Gruppe hinzufügen}{lst:ccjoingroup}
\begin{lstlisting}[float=h!t]
// Gruppe setzen
String groupURL = null;
switch (contactInfo.getType()) {
case CUSTOMER:
groupURL = customerGroupURL;
contact.addGroupMembershipInfo(new GroupMembershipInfo(false, groupURL));
\end{lstlisting}

\javalstset{Das Kontakt-Objekt (\lstinline{ContactEntry}) an den Service übergeben}{lst:ccsendcontact}
\begin{lstlisting}[float=h!t]
// Kontakt senden		
URL postUrl = new URL(contactsURL);
return myService.insert(postUrl, contact);
\end{lstlisting}
\FloatBarrier



\subsection{SIB-Programmierung}

In diesem Kapitel wird zunächst auf die Implementierung von SIBs im allgemeinen eingegangen.
Darauf folgt die genaue Beschreibung der für das Projekt erstellten SIBs, sowie die Darstellung zweier
 Besonderheiten im diesem Entwicklungsprozess.
\FloatBarrier
\subsubsection{Allgemeines zur SIB-Programmierung}
Im jABC können SIBs über eine Baumstruktur ausgewählt und dann bequem in das Prozessmodell eingebunden werden.
Um die verwendeten SIBs miteinander zu verbinden, besitzt jedes SIB fest definierte ausgehende Kanten, im
 jABC-Kontext auch Branches genannt. Zudem ist es üblich, dass ein SIB definierte Parameter benötigt, um dessen
 Ausführung zu steuern.
Je nach Zweck des einzelnen SIB unterscheiden sich daher dessen Branches und Parameter.
Im folgenden wird nun erläutert, wie ein solches SIB erstellt wird.

Da jABC eine Java-Anwendung ist und die Ausführung des Prozessmodells, und dadurch auch die Ausführung der
 einzelnen SIBs an sich, innerhalb von jABC stattfindet, werden die einzelnen SIBs ebenfalls in Java implementiert.
Ein SIB wird dabei durch eine Java-Klasse repräsentiert, welche mit der Annotation \lstinline{@SIBClass} versehen ist.
Durch diese Annotation wird die Java-Klasse von jABC als SIB erkannt.

Damit das SIB innerhalb einer Modellausführung verwendet werden kann, müssen jedoch zunächst noch Branches und
 Parameter definiert werden. Mögliche Branches werden dabei über eine öffentliche Klassenvariable namens \lstinline{BRANCHES}
 definiert, welche vom Typ \lstinline{String[]}-Array ist.
Auch SIB-Parameter sind öffentliche Klassenvariablen.
Der Name der Variablen ist beliebig und es kann aus den üblichen Standard-Typen (z.B. \lstinline{boolean}, \lstinline{int},
 \lstinline{String}, ...) ausgewählt werden.
 Für die Notwendigkeit der Anwendung von komplizierteren Parametern, wie ganzen Klassen, stellt jABC den Variablentyp
  \lstinline{ContextKey} zur Verfügung.
 Mit diesem ist es möglich Objekte innerhalb des Ausführungskontextes von jABC zu referenzieren.
 Diese können dann lesend wie schreibend verwendet werden.
 In jABC ist der Ausführungskontext über eine \lstinline{Map} realisiert.
 Daher auch der Name \lstinline{ContextKey} des Variablentyps, denn bei der Deklaration einer Variable von diesem Typ ist
  lediglich der zugehörige Schlüssel aus der \lstinline{Map} anzugeben.
 Die Klassenvariable kann dann dazu verwendet werden, auf die Daten im Ausführungskontext zuzugreifen.

Jedoch bedarf es noch einer weiteren Ergänzung, um ein SIB vollständig zu implementieren.
Es muss noch definiert werden, was dieses SIB eigentlich tun soll.
Hierfür muss die Java-Klasse des SIBs das Interface \lstinline{Executable} implementieren, welches verlangt,
 dass die Klasse die Methode \lstinline{trace()} definiert. Diese Methode \lstinline{trace()} wird innerhalb von jABC
 in dem Moment aufgerufen, wenn die Ausführung des SIBs beginnen soll.
Als Parameter wird eine Variable vom Typ \lstinline{ExecutionEnvironment} übergeben, welche für den Zugriff auf den
 Ausführungskontext erforderlich ist.
Zuletzt muss jABC noch mitgeteilt werden, welcher Branch im Anschluss an die Ausführung gewählt werden soll.
Die Methode \lstinline{trace()} besitzt hierzu den Rückgabewert \lstinline{String}.
Überlicherweise wird der Rückgabewert aus dem wie oben definierten \lstinline{String[]}-Array \lstinline{BRANCHES} ausgewählt.

Das \autoref{lst:samplesib} zeigt exemplarisch den Kopf einer typischen Implementierung einer SIB-Klasse.

% Listing SIB-Klasse %
\javalstset{MySIB}{lst:samplesib}
\begin{lstlisting}[float=h!t]
@SIBClass("My-SIB")
public class MySib implements Executable {

	// Branches
    public final String[] BRANCHES = {"default", "error"};

    // Parameter
    public \lstinline{ContextKey} someKey = new ContextKey("someKey");
    public String title = "Nice Title";

    @Override
    public String trace(ExecutionEnvironment env) {
        ...
		return "default";
    }
	...
}
\end{lstlisting}

% Listing SIB-Klasse ENDE %
\FloatBarrier
\subsubsection{Spezifikation notwendiger SIBs}
Wie bereits aus der Projektstruktur in Abbildung \ref{fig:projektaufbau} ersichtlich, sind verschiedene Arten von SIBs
 für ein geeignetes Modell notwendig.
Zum einen werden SIBs benötigt, welche die Funktionen der Konnektoren benutzen.
Zum anderen müssen SIBs implementiert werden, welche über GUI-Elemente die Eingaben des Nutzers abfragen.
Daraus ergibt sich die Notwendigkeit folgender SIBs.

Ein zentrales Element des Prozessen ist die Suche nach Kontakten anhand gegebener Kriterien.
Daher werden für die beiden Suchfunktionen (bei SAP und Google) passend zwei SIBs implementiert.
Die beiden SIBs unterscheiden sich jedoch nur in der Form, dass sie bei der Ausführung die jeweils passende Suchmethode
 der externen Datenbank verwenden.
Äußerlich unterscheiden sich diese, bis auf den Namen, daher nicht.
Als Eingabe erwarten beide Suchfunktionen einen Parameter vom Typ des in der Einleitung beschriebenen \lstinline{Contact}-Objektes.
Zurück liefern beide Funktionen eine Liste von eben genannten \lstinline{Contact}-Objekten. Parameter und Rückgabewert
 sind als SIB-Parameter vom Typ \lstinline{ContextKey} implementiert, wodurch eine weitere Verwendung möglich wird.
Definierte Branches der beiden SIBs sind \emph{found} (für den Fall dass die zurückgelieferte Liste nicht leer ist),
 \emph{not found} (im Falle einer leeren Rückgabeliste) und \emph{error} (wird gewählt, wenn die Ausführung der
 Suchfunktion eine \lstinline{Exception} wirft).\\

Ein weiteres Element einer Datenmigration ist das Hinzufügen der Objekte auf dem Zielsystem.
Zu diesem Zweck muss ein SIB erzeugt werden, welches die passende Funktion nutzt, um Kontakte dem Google-System hinzuzufügen.
Der Parameter des SIB ist in diesem Fall vom Typ \lstinline{ContextKey} und verweist auf ein \lstinline{Contact}-Objekt,
 dessen Variablen mit passenden Werten belegt sind.
Ein spezieller Rückgabewert ist in diesem Fall unnötig, denn Erfolg oder Misserfolg der Funktionsausführung wird über
 den gewählten Branch kommuniziert.
Zu diesem Zweck besitzt das SIB die zwei Branches \emph{default} (falls kein Fehler auftritt) und \emph{error} (falls
 \lstinline{Exception} geworfen wird, s.o.).

Damit ist die Kommunikation zu den externen Datenbanken ausreichen spezifiziert und es folgt die Definition der SIBs,
 welche für die Nutzereingaben verantwortlich sind.

Damit der Nutzer eine Suchanfrage definieren kann, muss jener die Daten eingeben können.
Hierzu wird ein SIB generiert, welches ein Fenster (Swing-Frame) öffnet und anschließend auf die Eingabe des Nutzers wartet.
Die Eingabefelder können mit den Werten eines \lstinline{Contact}-Objektes vorbelegt werden.
Wenn die Eingabe beendet ist, speichert das SIB die Daten in den Ausführungs"=kontext.
Zu diesem Zweck wurde ein passender SIB-Parameter vom Typ \lstinline{ContextKey} definiert.
Dieser dient sowohl für die Vorbelegung, als auch zur Speicherung der Eingaben.
Um die Wiederverwendbarkeit zu ermöglichen, wurde das SIB um weitere Parameter erweitert.
Zum einen besitzt das \lstinline{Contact}-Objekt eine Funktion namens \lstinline{validate()}, mit Hilfe derer sich die
 Eingaben des Nutzers kontrollieren lassen.
Diese kann mittels eines Parameters vom Typ \lstinline{boolean} an- bzw. ausgeschaltet werden.
Auf diese Weise ist es möglich das SIB sowohl für die Suchanfrage (keine Validierung notwendig) als auch zur Dateneingabe
 eines neuen Kontakts (Validierung notwendig) zu verwenden.
Zudem wurden die Branches \emph{ok} (Eingabe beendet), \emph{cancel} (Eingabe abgebrochen) und \emph{error} (s.o.) definiert.

\begin{figure}[h!t]
\includegraphics[width=\textwidth]{Bilder/Sib_EditContact_Frame.png}
\caption{Kontaktdaten anzeigen und verändern}
\label{fig:sibedit}
\end{figure}

Zuletzt fehlt noch die Möglichkeit, dass der Nutzer einen Kontakt aus einer Liste von Kontakten auswählen kann,
 sprich auf das Ergebnis einer Suchanfrage reagieren kann.
Das hier zu implementierende SIB benötigt also einen Parameter in Form einer Liste von \lstinline{Contact}-Objekten.
Auf der anderen Seite muss das \lstinline{Contact}-Objekt zurückgeliefert werden, welches der Nutzer ausgewählt hat.
Ähnlich zu vorherigen SIBs, sind diese Parameter als \lstinline{ContextKey} realisiert.
Das SIB liest die Kontaktliste aus dem Ausführungskontext und befüllt ein entsprechendes Fenster.
Nachdem der Nutzer einen Kontakt ausgewählt hat, werden die Daten entsprechend in den Kontext geschrieben.
Es wurden die Branches \emph{ok} (Eingabe beendet), \emph{cancel} (Eingabe abgebrochen) und \emph{error} (s.o.) definiert.

%Abbildung Choose
%\usepackage{graphics} is needed for \includegraphics
\begin{figure}[h!t]
  \includegraphics[width=\textwidth]{Bilder/Sib_ChooseContact_Frame.png}
  \caption{Einen Kontakt aus der Liste der Suchergebnisse auswählen}
  \label{fig:sibchoose}
\end{figure}

Um der Konvention von jABC zu folgen, wurde noch ein weiteres SIB implementiert.
In jABC ist es üblich ein sogenanntes \lstinline{Put}-SIB zu verwenden, wenn eine Variable eines bestimmten Typs im
 Ausführungskontext erzeugt wird.
Zu diesem Zweck startet das Prozessmodell mit einem SIB namens \lstinline{PutContact}.
Es erzeugt eine Instanz der \lstinline{Contact}-Klasse im Ausführungskontext, welche im Verlauf der Modellausführung
 verändert wird, bis ihr Inhalt zum Schluss dem Google-System hinzugefügt wird.
	
\FloatBarrier
\subsubsection{Besonderheit der GUI-SIBs}
Eine Besonderheit bei der Implementierung von SIBs ist die Benutzung von Komponenten des Swing-Frameworks.
Üblicherweise werden die erzeugten Fenster in einem separaten Thread gestartet, was den Effekt zur Folge hat, dass nach
 Erzeugung des Fensters die \lstinline{trace()}-Methode nicht auf Eingaben wartet, sondern weiter ausgeführt wird.
Während der Nutzer seine Eingabe noch nicht einmal begonnen hat, ist jABC mit der Ausführung der \lstinline{trace()}-Methode
 schon fertig.

Als Lösung des Problems bot sich die Java-eigene Möglichkeit zur Thread-Synchronisation mittels eines synchronized-Blockes an.
Eine exemplarische Verwendung zeigt \autoref{lst:sibsynchro}.

\javalstset{...}{lst:sibsynchro}
\begin{lstlisting}[float=h!t]

\end{lstlisting}
% synchro listing %

In diesem Listing ist der synchronized-Block der \lstinline{trace()}-Methode zu sehen.
Nachdem der Swing-Frame erzeugt wurde, dient jener als Synchronisationsobjekt.
Nachdem der \lstinline{synchronized}-Block betreten wurde, wird in \lstinline{trace()} die Methode \lstinline{wait()} aufgerufen.
An dieser Stelle stoppt die Ausführung von \lstinline{trace()}, bis das entsprechende Gegenstück, die Methode \lstinline{notify()},
 auf dem Synchronisationsobjekt aufgerufen wurde.
Dies geschieht innerhalb der ActionListener der verwendeten Buttons.
Wenn der Nutzer mit der Eingabe der Daten oder der Auswahl des Kontakts fertig ist, signalisiert er dies mit einem
 Klick auf einen Button.
Es wird der entsprechende \lstinline{ActionListener} des Buttons aufgerufen, welcher wiederum innerhalb eines
 \lstinline{synchronized}-Block die Methode \lstinline{notify()} aufruft und anschließend das Fenster schließt.

Die wartende \lstinline{trace()}-Methode wird daraufhin weiter ausgeführt und kann über geeignete Klassenvariablen
 des Frame-Objektes die Eingaben des Nutzers abfragen und in den Ausführungskontext schreiben.
\FloatBarrier
\subsubsection{Die JAVA-Laufzeitumgebung in jABC}	
Eine weitere Besonderheit hat sich während des Projektes zufällig ergeben.
Um das Zusammenspiel von Konnektoren und GUI-Elementen effizient zu testen, wurde auf die Ausführung der
 Komponenten im jABC zunächst verzichtet.
Erst zum Ende des Projektes wurden die Komponenten mittels ihrer zugehörigen SIBs in jABC getestet.
Es zeigte sich ein Fehler während der Ausführung, welcher außerhalb von jABC nicht aufrat.
Die geworfene \lstinline{Exception} ließ vermuten, dass bestimmte Bibliotheken innerhalb der Ausführung von jABC nicht
 gefunden werden konnten.
Unter Verwendung des selben Quellcodes außerhalb von jABC zeigte sich dieser Fehler jedoch nicht.
Dies legt die Vermutung nahe, dass die zur Laufzeit existierende Java-Umgebung in Form der Java-Virtual-Maschine (JVM)
 in beiden Szenarien nicht gleich ist.
Einige Bibliotheken werden innerhalb von jABC unter anderen Paketpfaden referenziert, als es außerhalb der Fall ist.
Zu diesem Zweck mussten die Eigenschaften der JVM im jABC manuell verändert werden. Dies ist in Listing \ref{lst:sibworkaround}
 zu sehen.

\javalstset{Workaround}{lst:sibworkaround} 
\begin{lstlisting}[float=h!t]

\end{lstlisting}

% Listing system.setproperty %

Auch wenn die Veränderung der Laufzeit-Eigenschaften das Problem behoben haben, so ist doch generell von dieser Praxis abzuraten.
Denn die veränderten Eigenschaften bleiben auch noch nach der Modellausführung im jABC bestehen, da jABC selbst ein Java-Programm
 ist und zur Ausführung von Modell die JVM des Programms verwendet wird.
Die Ausführung anderer Modelle im Anschluss an jenes dieses Projektes kann also beeinträchtigt oder im schlimmsten Fall gar
 nicht erst möglich sein.
\FloatBarrier

\subsection{Das Modell im jABC}


Abschließend fließen nun die einzelnen Komponenten, welche durch entsprechende SIBs repräsentiert werden, in das Prozessmodell ein. Die folgende Abbildung \ref{fig:jabcmodel} zeigt das fertige Projektergebnis, welches die in Kapitel \ref{aufgabe} beschriebene Aufgabenstellung erfüllt.\\

\begin{figure}[h!t]
\begin{center}
\includegraphics[width=0.5\textwidth]{Bilder/jabc_Model.png}
\end{center}
\caption{fertiges Prozessmodell im jABC}
\label{fig:jabcmodel} 
\end{figure} 

Zunächst lassen sich die farblich-gekennzeichneten Bereiche erkennen. Die blauen Bereiche enthalten die Kommunikation mit der Datenbank bei Google, während der gelbe Bereich die Kommunikation mit dem SAP-System repräsentiert. Die farbige Kennzeichnung dient auch der Tatsache, dass sich diese Bereiche in Untermodelle auslagern lassen, um diese z.B. in anderen Modellen wiederzuverwenden. Im Rahmen dieses Projektes wurde jedoch aus Gründen der Übersichtlichkeit darauf verzichtet. Ebenso verhält es sich mit den in Kapitel \ref{sibs_spez} beschriebenen \textit{error}-Branches. \\
Zudem ist die Wiederverwendung der SIBs erkennbar, welche für die Nutzereingabe verantwortlich sind und es fällt auf, dass die Struktur der Suchanfragen bei SAP und Google eine gemeinsame Struktur haben.






\subsection{Zusammenfassung}
\section{Beurteilung des Projektes}
\label{sec:beurteilung}

Zusammenfassend lässt sich das Projektergebnis als Erfolg verzeichnen. Zum einen natürlich, weil das erzeugte jABC-Modell den Anforderungen der Aufgabenstellung genügt. Der Großteil des Erfolges ist jedoch sicherlich der Erfahrung geschuldet, welche während des Projektes gesammelt werden konnte. So zeigten sich während der Bearbeitung die Besonderheiten einer Datenmigration in Zusammenhang mit der Verwendung von Webservices. 

So muss zum einen darauf geachtet werden, ob gewählte Webservices überhaupt die Methoden zu Verfügung stellen, welche für die Aufgabenstellung benötigt werden. Während es in diesem Fall zum Beispiel möglich war bei SAP nach einzelnen Attributen von Personen zu suchen, bot der Webservice von Google diese Möglichkeit nicht. Hier mussten bei jeder Anfrage alle Kontakte zurück"=geliefert werden. Ein Aussortieren fand erst im Anschluss lokal statt.

Zum anderen ist darauf zu achten, dass im Zielsystem auch alle Attribute des Quellsystems gespeichert werden können. Unbedingt muss in diesem Zusammenhang auch darauf geachtet werden, eine ausreichende Zuordnung der Attribute des Quell- und Zielsystems zu spezifizieren. In einigen Fällen kann es daher vorkommen, dass sich bestimmte Informationen bei einer Migration nicht übertragen lassen. Im speziellen Fall dieses Projektes gab es hinsichtlich dieses Aspektes keine Probleme.

Nicht zuletzt kann das Projekt auch hinsichtlich der Zusammenarbeit der Projektteilnehmer als Erfolg verzeichnet werden. 





\clearpage
\section{Literaturverweise}
	\begin{thebibliography}{mustermarke}
	
		\bibitem{WF95} Wasserman, Stanley und Katherine Faust; \textit{Social Network Analysis: Methods and Applications}; Cambridge University Press, New York; 1995
	
		\bibitem{CLRS01} Cormen, Thomas H., Charles E. Leiserson, Ronald L. Rivest und Clifford Stein; \textit{Introduction to Algorithms, Second Edition}; The MIT Press and McGraw-Hill Book Company; 2001
		
			\bibitem{JKL04} Jacob, Riko, Dirk Koschtzki, Katharina Anna Lehmann, Leon Peeters und Dagmar Tenfelde-Podehl; \textit{Algorithms for Centrality Indices};\\
		In: Brandes, Ulrik und Thomas Erlebach
	
		\bibitem{BE05} Brandes, Ulrik und Thomas Erlebach; \textit{Network Analysis: Methodological Foundations}; Band 3418 der Reihe \textit{Lecture Notes in Computer Science}; Springer; 2005
	
	\end{thebibliography}
\end{document}
