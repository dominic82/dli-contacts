\section{Einleitung}

- Im Rahmen der Vorlesung DLI ... Dieser Bericht erläutert im Detail ...

- In diesem Kapitel
	- Zunächst Aufgabe darstellen
	- Daraufhin Skizzierung der Projektstruktur und erste allgemeine Überlegungen

- In Kapitel 2 Darstellung der einzelnen Projektbausteine im Detail: 
	- Beginnt mit den Connectoren: SAP-Connector, Google-Connector
	- Dann Bereich der SIB-Programmierung: im Allgemeinen und im Speziellen bezogen auf das Projekt
	
- In Kapitel 3 Darstellung des fertigen jABC-Modells
	- Erklärung Modellaufbau
	- Erklärung von Design-Entscheidungen (Meta-SIBs), Error-Branches
	
- Abschließend in Kapitel 4 eine kurze Zusammenfassung sowie Beurteilung über den Erfolg des Projektes	


\subsection{Projektbeschreibung}
- fiktives Szenario: Umstellung der informationstechnischen Infrastruktur bei einem Unternehmen
	- In diesem Projekt der Teil, welcher die Migration von Stammdaten der Kontakte des Unternehmens betrifft
		- Kontakte gliedern sich in drei Gruppen: Kunde, Lieferant und Angestellter
	- Migration erfolgt von SAP nach Google. Dabei kommt nicht-triviale Strategie zum Einsatz (Karteilleichen):
		- Kontakte nur nach Bedarf übertragen
		- Erst Google prüfen, dann versuchte Migration von SAP, sonst ganz neuen Datensatz bei Google anlegen

\subsection{Erste Überlegungen}
- Als erstes Zergliederung des Projektes, Aufgabenverteilung, Berührungspunkte, eingesetzte Technologien

\subsubsection{Projektstruktur}
- Analyse der Projektstruktur / Identifikation von autonomen Mechanismen:
	- zur Aufgabenverteilung an einzelne Mitglieder: Spezialisten schaffen, Verantwortungsbereiche festlegen
	- Identifikation von Berührungspunkten der Teilbereiche
	- Aufteilung des Projektes (siehe Grafik)
	
\subsubsection{Das Kontakt-Objekt als Schnittstelle}
- Das Kontakt Objekt:
	- Bildung einer repräsentativen Schnittmenge von typischen Kontaktattributen
		- Wahl der Attribute
		- ID-Speicherung für Möglichkeit der Anwendung von komplexeren Migrations-Strategien
	- Modellierung einer allgemeinen Klasse Kontakt als Schnittstellenobjekt
		- zwischen den Funktionen, innerhalb des jABC-Modells
		
\subsubsection{Eingesetzte Technologien}
- Verteilte Programmierung -> VCS git
- verwendete Sprache Java -> JUnit-Tests zur Verifikation der Schnittstelle zw. SIBs und Konnektoren
- Apache-Maven im Hinblick auf Unterstützung von build und deploy Aufgaben
	- autom. Tests im build
	- unterstützung beim deploy durch erzeugung eines jar-Archivs inkl. notwendiger Abhängigkeiten
		- eingentlich Bad Practice
		- Im Umgang mit jABC simpler


\section{Unsere Projektbausteine}
 - Im Folgenden Beschreibung, besondere Überlegungen und Probleme(!)/Entscheidungen der einzelnen Elemente

