\subsubsection{Die Bibliothek \emph{gdata}}
Die Google-Bibliothek \emph{gdata} ist eine frei verf\"ugbare Bibliothek zum erstellen von
 Clientapplications für die Services der Google-Cloud.
\emph{gdata} kapselt die Webservices komplett in Java-Klassen, so dass ein importieren
 (z.\ B.\ mit \emph{wsimport}) nicht mehr notwendig ist.

In der Beschreibung der \emph{gdata}-Bibliothek wird beschrieben, wie man aus der zip-Datei
 ein JAR compilieren kann, da dies bei mir in mehreren Versuchen nicht geklappt hat,
 haben wir die pragmatische Lösung gewählt und die in der zip-Datei enthaltenen JARs von Hand
 in das Projekt eingefügt.
%TODO Link zu gdata

\subsubsection{Authentifizieren und Verbinden mit \emph{gdata}}
Google bietet zwei Authentifizierungsverfahren an
\begin{enumerate}
	\item\emph{OAuth}
	\item Username und Passwort
\end{enumerate}
\emph{OAuth} ist ein Service, der bei erfolgreicher Anmeldung ein Token erstellt, mit dem
 der Client von Google bereitgestellte Services aufrufen und sich authentifizieren kann.
So muss der Client die Anmelde-Daten des Nutzers nicht speichern, sondern nur den Token.
In Abbildung \ref{fig:google_oauth} wird ein Beispiel f\"ur die Nutzung von \emph{OAuth} dargestellt.
\begin{figure}[h!]
\includegraphics[width=\textwidth]{Bilder/googleOauth.png}
\caption{Nutzungsbeispiel f\"ur \emph{OAuth}\cite{GO01}}
\label{fig:google_oauth}
\end{figure}

Da wir unseren eigenen Account nutzen und die Daten nicht sicherheitskritisch sind, haben
 wir die zweite Variante gew\"ahlt und authentifizieren uns bei jedem Service-Aufruf mit
 Username und Passwort.

Die Authentifizierung wird für ein \emph{ContacsService}-Objekt, wie im Codeschnipsel
 \ref{lst:authpwbeispiel} abgebildet, einmal durchgeführt, danach wird sie vom Framework
 automatisch durchgeführt.

\lstset{% general command to set parameter(s)
	basicstyle=\small\ttfamily, % print whole listing small
	keywordstyle=\color{DarkOrchid}\bfseries,
	% underlined bold black keywords
	identifierstyle=, % nothing happens
	commentstyle=\color{Gray}, % white comments
	stringstyle=\color{Blue}, % typewriter type for strings
	showstringspaces=false % no special string spaces 
	tabsize=3,
	language=Java,
	caption={Beispiel für die Authentifizierung ohne \emph{OAuth}},
	label=lst:authpwbeispiel,
	breaklines=true,
	breakatwhitespace=true}
\begin{lstlisting}
ContactsService myService;
myService = new ContactsService(servicename);
try {
	myService.setUserCredentials(username, password);
} catch (AuthenticationException e) {
	e.printStackTrace();
}
\end{lstlisting}

\subsubsection{Kontakte suchen}
Die \emph{gdata}-Bibliothek bietet die Möglichkeit, Kontakte mit Angabe eines \emph{Querys}
 herunterzuladen.
Das \emph{Query} kann jedoch nur zwischen Gruppen unterscheiden, jedoch nicht nach anderen
 Kriterien wie z.\ B.\ dem Vornamen oder dem Nachnamen eines Kontakts filtern.

Das Suchen von Kontakten geschieht in unserem Projekt durch das Herunterladen aller Kontakte
 einer Gruppe und anschließendem sortieren "`von Hand"'.

\subsubsection{Kontakte einf\"ugen}
Das Einfügen von Kontakten ist über ein erstelltes Service-Objekt 