Generelles zum Aufbau von SIBs. Besonderheiten bezogen auf das Projekt

\subsubsection{Allgemeines zur SIB-Programmierung}
- SIBs als Java-Klasse mit Annotation
- Muss best. Interface implementieren
	- eine Funktion trace, welche innerhalb von jABC bei der Ausführung aufgerufen wird
	- Übergabeparamter ist der Ausführungskontext (MAP) mit Laufzeitdaten
	
\subsubsection{Spezifiktion notwendiger SIBs}	
- 3 Sorten von SIBS
	- Google SIBs, SAP-SIBs, GUI-SIBs
	- Skizzierung von Search, Add, Edit, Choose
	- sonderfall "PutContact"
- Zusammenfassend 6 SIBs:

\subsubsection{Besonderheit der GUI-SIBs}
- synchronisation mir Nutzereingabe
- Abfrage der Nutzereingaben im SIB

\subsubsection{die JAVA-Laufzeitumgebung in jABC}	
- siehe Folie

