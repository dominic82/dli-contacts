\subsubsection{Aufgabe des SAP Connectors}

Die Aufgabe des SAP Connectors besteht darin, die Datenbank im ES Workplace nach einem 
gegebenen Filter zu durchsuchen. Die gefilterten Datensätze werden als Liste zurückgegeben. 

\subsubsection{Aufbau und Programmablauf}

Alle Operationen dienen dazu, Datensätze aus den Datenbanken des ES Workplace auszulesen. Der ES Workplace stellt hierfür 
eine Datenbank mit Testdatensätzen bereit, sowie die erforderlichen Webservices, um eine Verbindung zur Datenbank, 
die Authentifizierung und die Suche selbst durchzuführen. Die erforderlichen Webservices liegen als XML Dateien vor, werden 
jedoch zur Verwendung in SIBs als Java Klassen benötigt. Dafür wird Java API for XML Web Services verwendet, welches seit
Version 1.6  Bestandteil der Java SE ist. Mittels Jax-WS können in der Konsole durch einen wsimport Befehl alle benötigten 
Webservices von XML Dateien in Java Klassen umgewandelt werden. 

Der SAP Connektor erhält ein Objekt vom Typ Contact. Dieses Objekt enthält die vom Benutzer geforderten Filtereinstellungen 
betreffend Vorname, Nachname, Postleitzahl, Stadt, Straße und Firma. Außerdem wird der Typ nach Kunde, Lieferant und 
Mitarbeiter unterschieden. Dieses Objekt wird verwendet, um in der Datenbank des ES Workplace, nach dem entsprechenden Typ,
Datensätze zu filtern. In Abhängigkeit von der Typbestimmung werden unterschiedliche Webservices ausgeführt:

\textbf{Lieferant}
\begin{itemize}
\item Find Supplier by Name and Address
\item Read Supplier Basic Data 
\end{itemize}

\textbf{Mitarbeiter}
\begin{itemize}
\item Find Employee by Elements
\item Find Employee Address by Employee
\end{itemize}

\textbf{Kunde}
\begin{itemize}
\item Find Customer by Elements
\end{itemize}

Zunächst soll hier das Programm am Beispiel des Kundenwebservices beschrieben werden. Anschließend werden die Besonderheiten
bei den Webservices für Mitarbeiter und Lieferant erläutert.

Zunächst müssen die notwendigen Objekte der Klassen des Webservices Find Customer by Elements erstellt werden. Mittels
einer Klasse die immer mit dem Schlüsselwort SERVICE beginnt, kann ein Objekt erzeugt werden auf dem eine getBinding Methode
ausgeführt wird. Das erstellte Verbindungsobjekt wird dann zu einem Objekt vom Typ Binding Provider gecastet, auf welchem dann
mittels getRequestContext() Methoden Passwort und Username gesetzt werden können. Für die Authentifizierung ist normalerweise
auch noch eine Webadresse anzugeben, allerdings ist hier die Adresse der Testdatenbank bereits enthalten. Das 
Verbindungsobjekt verfügt über eine einzige Methode, die ein Objekt mit Suchkriterien übergeben bekommt und eine Liste 
von Suchergebnissen zurückliefert. Die Suchkriterien können mit bereits integrierten set Methoden gesetzt werden,
die Suchergebnisse über get Methoden ausgelesen werden. Anschließend werden die Suchergebnisse als Liste von Kontaktobjekten
zurückgegeben.

Die Webservices Find Supplier by Name and Address und Find Employee by Elements funktionieren ähnlich, liefern allerdings nur 
einle Liste mit Namen und SAP-IDs. Die SAP-IDs sind eindeutige Nummern, die jedem Datensatz zugeordnet werden. Den Webservices
Read Supplier Basic Data und Find Employee Address by Employee ist es möglich die fehlenden Adressdaten mit den SAP-IDs 
auszulesen. Dies muss jedoch für jede SAP-ID separat geschehen, also muss für jede SAP-ID von Lieferant und Mitarbeiter der
entsprechende Webservice einzeln aufgerufen werden, die Übergabe einer Liste von IDs ist nicht vorgesehen. Danach können 
die Suchergebnisse auch hier als Liste von Kontaktobjekten zurückgegeben werden.


\subsubsection{Probleme}

Die Verwendung der Webservices des ES Workplace bringt einige Herausforderungen mit sich was Programmierung und Anwendung
angeht. Die Programmierung wird durch die hohe Anzahl unterschiedlicher Klassen in den Webservices erschwert. Da für jeden
Webservice eigene Objekte erstellt werden müssen, die in den anderen Webservices, selbst bei gleichem Inhalt wie PLZ oder
Stadtname, nicht wiederverwendet werden können, ist eine prozeduraler Programmieransatz die beste Lösung. Eine objektorientierte 
Lösung ist mittels vieler Interfaces, sehr guter Planung und hohem Entwicklungsaufwand zwar auch realisierbar, ob die leicht 
verbesserte Wiederverwendbarkeit den Aufwand allerdings rechtfertigt ist fraglich. Zumal lediglich der Verbindungsaufbau in allen
WSDLs als ähnlich zu betrachten ist, selbst get und set Methoden sind völlig unterschiedlich. Zum Beispiel müssen im Mitarbeiter
Webservice erst drei Objekte der inneren Klassen erstellt werden, um in der ``innerstenn'' Objekt einen String zuzuweisen und 
dann die Objekte den Objekten der äußeren Klassen zuzuweisen:

CustomerSelectionByNameAndAddress kundeFilter 	= new CustomerSelectionByNameAndAddress();
AddressInformation add1							= new AddressInformation();
Address add2 									= new Address();
PhysicalAddress add3 							= new PhysicalAddress();

add3.setCityName(``Hamburg'');
add2.setPhysicalAddress(add3);
add1.setAddress(add2);

kundeFilter.setAddressInformation(add1);
 

Ein weiteres Problem stellen die separaten Einzelaufrufe der Webservices für jede ID dar. 

\subsubsection{Ausblick}