%%%%%%%%%%%%%%%%%%%%%%%%%%
% BITTE IN UTF-8 OEFFNEN %
%%%%%%%%%%%%%%%%%%%%%%%%%%
\documentclass[xcolor={usenames,dvipsnames}, compress, 10pt]{beamer}

\usepackage[ngerman] {babel}
\usepackage[utf8] {inputenc}

\usepackage{tikz}
\usepackage{graphics}
\usepackage{BeamerColor}
\usepackage{expdlist} 
\usepackage{eurosym}

\usetheme{Warsaw}
\usefonttheme{professionalfonts}

\usecolortheme[named=ForestGreen]{structure}

\usepackage{ownTheme}
\setbeamertemplate{footline}[infolines theme]
\setbeamertemplate{headline}[miniframes theme]
\setbeamercovered{transparent=8}

%Code
\usepackage{listings}
%fuer fette typewriter Schrift
\renewcommand{\ttdefault}{pcr}

%Farben (benoetigt fuer listings)
\usepackage{color}
%\usepackage[usenames,dvipsnames]{xcolor}

%vor jeder lstlisting-Umgebung aufrufen, 1. caption, 2. label
\newcommand{\javalstset}[2]{
\lstset{% general command to set parameter(s)
	basicstyle=\small\ttfamily, % print whole listing small
	keywordstyle=\color{DarkOrchid}\bfseries,
	% underlined bold black keywords
	identifierstyle=, % nothing happens
	commentstyle=\color{Gray}, % white comments
	stringstyle=\color{Blue}, % typewriter type for strings
	showstringspaces=false % no special string spaces 
	tabsize=3,
	language=Java,
	captionpos=b,
	caption={#1},
	label={#2},
	frame=trbl,
	breaklines=true,
	breakatwhitespace=true
	}
}

\AtBeginSection[]
{
\begin{frame}
\frametitle{}
\tableofcontents[currentsection]
\end{frame}
}

\begin{document}

\title[DLI Projekt: Kontakte]{DLI Projekt: Kontakte}
\subtitle[DLI]{Vorlesung Aktuelle Themen der Dienstleistungsinformatik}
%\date{\today} % gibt aktuelles Datum zurueck
\date{22. Januar 2013}
\author[M.\ Marzotko, T.\ Seeland, D.\ Wirkner]{Markus Marzotko, Thorben Seeland \& Dominic Wirkner\\ {\scriptsize Prof.\ Dr.\ Bernhard Steffen, Dipl.-Inf. Markus Doedt}} 
\institute[TU Dortmund]{Technische Universit\"at Dortmund}

\nocite{*}

\frame{\titlepage}

\begin{frame}[fragile]{Test}
\javalstset{Den Kontakt einer Gruppe hinzufügen}{lst:ccjoingroup}
\begin{lstlisting}
// Gruppe setzen
String groupURL = null;
switch (contactInfoCopy.getType()) {
case CUSTOMER:
groupURL = customerGroupURL;
contact.addGroupMembershipInfo(new GroupMembershipInfo(false, groupURL));
\end{lstlisting}
\end{frame}

\begin{frame}
%%%%%%%%%%%%%%	Gliederung		%%%%%%%%%%%%%%
\tableofcontents
\end{frame}			

\section{Einf\"uhrung}

\subsection*{Begriffskl\"arung}

\begin{frame}{Was ist Compliance?}
\pause
\begin{block}{Definition 1}
		\begin{quote}
			"`Die Binsenweisheit, dass Unternehmen Gesetze einhalten m\"ussen, hei{\ss}t nun
			Compliance."'
		\end{quote}
			Uwe H. Schneider, ZIP 2003, S. 645f
\end{block}
\pause
\begin{block}{Definition 2}
	\begin{quote}
		Bei "`Compliance"' geht es um die "`Erf\"ullung"', "`Entsprechung"' bzw.
		"`Konformit\"at"' mit staatlichen Gesetzen sowie mit Regeln und Spezifikationen,
		mit Grunds\"atzen (ethische und moralische) und Verfahren sowie mit Standards
		(z.B. ISO) und Konventionen, die klar definiert worden sind. Die Erf\"ullung der
		Compliance kann sowohl auf Zwang (z.B. durch Gesetze) als auch auf
		Freiwilligkeit (z.B. Einhaltung von Standards) beruhen.
	\end{quote}
		laut Compliance-Magazin.de
\end{block}
\end{frame}

\subsection*{Typische Anwendungsfelder}

\begin{frame}{Typische Anwendungsfelder}
\pause
\begin{itemize}[<+->]
	\item Medizin
		\begin{itemize}[<+->]
			\item ordnungsgem\"a{\ss}es Verschreiben von Medikamenten
			\item Befolgen der Ratschl\"age von \"Arzten
			\item Physiologie: Dehnbarkeit von K\"orperstrukturen
		\end{itemize}
	\item Cross-Compliance
		\begin{itemize}[<+->]
			\item landwirtschaftlicher Kontext
			\item betrifft Verpflichtungen f\"ur den Erhalt von EU-Direktbeihilfen:
				\begin{itemize}[<+->]
					\item landwirtschaftlicher und \"okologischer Zustand der Fl\"achen
					\item Futtermittel und Lebensmittelsicherheit
					\item Tiergesundheit und Tierschutz
				\end{itemize}
		\end{itemize}
	\item IT-Compliance
	\begin{itemize}[<+->]
		\item "`unsere"' Compliance
		\item Bedeutungsgewinn durch Digitalisierung von Gesch\"aftsprozessen
		\item Gesetze und Standards als Reaktion auf Computerpannen etc.
	\end{itemize}
\end{itemize}
\end{frame}

\section{Rechtliche Grundlagen}
\subsection*{Sarbanes-Oxley Act}

\begin{frame}{Enron, Worldcom und Neuer Markt}
\begin{itemize}
	\pause
	\item \textbf{Enron} (USA, Energiekonzern)
	\pause
	\item \textbf{Worldcom} (USA, Telekommunikationskonzern)
	\pause
	\item \textbf{Neuer Markt}
\end{itemize}
\end{frame}

%%% Einführung %%%
\section{Einleitung}

- Im Rahmen der Vorlesung DLI ... Dieser Bericht erläutert im Detail ...

- In diesem Kapitel
	- Zunächst Aufgabe darstellen
	- Daraufhin Skizzierung der Projektstruktur und erste allgemeine Überlegungen

- In Kapitel 2 Darstellung der einzelnen Projektbausteine im Detail: 
	- Beginnt mit den Connectoren: SAP-Connector, Google-Connector
	- Dann Bereich der SIB-Programmierung: im Allgemeinen und im Speziellen bezogen auf das Projekt
	
- In Kapitel 3 Darstellung des fertigen jABC-Modells
	- Erklärung Modellaufbau
	- Erklärung von Design-Entscheidungen (Meta-SIBs), Error-Branches
	
- Abschließend in Kapitel 4 eine kurze Zusammenfassung sowie Beurteilung über den Erfolg des Projektes	


\subsection{Projektbeschreibung}
- fiktives Szenario: Umstellung der informationstechnischen Infrastruktur bei einem Unternehmen
	- In diesem Projekt der Teil, welcher die Migration von Stammdaten der Kontakte des Unternehmens betrifft
		- Kontakte gliedern sich in drei Gruppen: Kunde, Lieferant und Angestellter
	- Migration erfolgt von SAP nach Google. Dabei kommt nicht-triviale Strategie zum Einsatz (Karteilleichen):
		- Kontakte nur nach Bedarf übertragen
		- Erst Google prüfen, dann versuchte Migration von SAP, sonst ganz neuen Datensatz bei Google anlegen

\subsection{Erste Überlegungen}
- Als erstes Zergliederung des Projektes, Aufgabenverteilung, Berührungspunkte, eingesetzte Technologien

\subsubsection{Projektstruktur}
- Analyse der Projektstruktur / Identifikation von autonomen Mechanismen:
	- zur Aufgabenverteilung an einzelne Mitglieder: Spezialisten schaffen, Verantwortungsbereiche festlegen
	- Identifikation von Berührungspunkten der Teilbereiche
	- Aufteilung des Projektes (siehe Grafik)
	
\subsubsection{Das Kontakt-Objekt als Schnittstelle}
- Das Kontakt Objekt:
	- Bildung einer repräsentativen Schnittmenge von typischen Kontaktattributen
		- Wahl der Attribute
		- ID-Speicherung für Möglichkeit der Anwendung von komplexeren Migrations-Strategien
	- Modellierung einer allgemeinen Klasse Kontakt als Schnittstellenobjekt
		- zwischen den Funktionen, innerhalb des jABC-Modells
		
\subsubsection{Eingesetzte Technologien}
- Verteilte Programmierung -> VCS git
- verwendete Sprache Java -> JUnit-Tests zur Verifikation der Schnittstelle zw. SIBs und Konnektoren
- Apache-Maven im Hinblick auf Unterstützung von build und deploy Aufgaben
	- autom. Tests im build
	- unterstützung beim deploy durch erzeugung eines jar-Archivs inkl. notwendiger Abhängigkeiten
		- eingentlich Bad Practice
		- Im Umgang mit jABC simpler


\section{Unsere Projektbausteine}
 - Im Folgenden Beschreibung, besondere Überlegungen und Probleme(!)/Entscheidungen der einzelnen Elemente



%SAP 
\section{SAP Connector}

\subsection*{}

\begin{frame}{Aufgabe}
\begin{center}

\begin{itemize}
\item Erhalte Kontaktobjekt mit Angaben zu Typ, Vorname, Nachname, Firma..
\item Filtere aus Datenbank entsprechende Datensätze
\item Gib aufbereitete Liste aller zutreffenden Kontakte zurück
\end{itemize}

\end{center}
\end{frame}

\subsection*{}

\begin{frame}{Verwendete WSDLs}
\begin{center}

\textbf{Lieferant}
\begin{itemize}
\item Find Supplier by Name and Address
\item Read Supplier Basic Data 
\end{itemize}

\textbf{Mitarbeiter}
\begin{itemize}
\item Find Employee by Elements
\item Find Employee Address by Employee
\end{itemize}

\textbf{Kunde}
\begin{itemize}
\item Find Customer by Elements
\end{itemize}



\end{center}
\end{frame}

\subsection*{}

\begin{frame}{Programmablauf}
\begin{center}

\begin{itemize}
\item Art des Filterobjekts überprüfen und entsprechenden Webservice aufrufen
\item IDs auslesen und anderen Webservice für alle IDs (einzeln) aufrufen
\item Rückgabeobjekte auslesen und Daten geordnet zurückgeben
\item[$\rightarrow$]Bei Kunde reicht ein Webserviceaufruf
\end{itemize}

\end{center}
\end{frame}

\subsection*{}

\begin{frame}{Lieferant, Kunde und Mitarbeiter: 3 unterschiedliche Welten}
\begin{center}

\begin{itemize}
\item Lieferant, Kunde und Mitarbeiter verwenden völlig unterschiedliche Klassen
\item lediglich die Übergabe von Passwort und Username ist gleich
\item Employee kommt aus dem SAP Human Ressources (HR) Bereich
\item [$\rightarrow$]Selbst einfachste Zuweisungen verkommen hier zur Akkordarbeit
\end{itemize}

\end{center}
\end{frame}

\subsection*{}

\begin{frame}{Problematik bei Lieferant und Kunde}
\begin{center}

\begin{itemize}
\item Webservices geben hier nur Liste von IDs und Namen zurück
\item Für Adressinformationen weiterer Aufruf mit anderem Webservice nötig
\item \textbf{PROBLEM: } Aufruf geschieht für jede ID einzeln
\end{itemize}

\end{center}
\end{frame}

\subsection*{}

\begin{frame}{Problemdarstellung}
\begin{center}

\includegraphics[width=\textheight]{Bilder/presi1.png} 

\end{center}
\end{frame}

\subsection*{}

\begin{frame}{Lösung}
\begin{center}

\begin{itemize}
\item GUI so erstellen, dass zunächst nur Name/Firma angezeigt werden
\item Name/Firma sind bereits nach erstem Webservice Aufruf vorhanden
\item Erst nach Klick auf Namen werden Adressdaten via Webservice angefordert
\end{itemize}

\end{center}
\end{frame}

\subsection*{}

\begin{frame}{Lösung}
\begin{center}


\includegraphics[width=\textheight]{Bilder/presi2.jpg} 

\end{center}
\end{frame}

%%% jABC Teil Dominic %%%
\section{SIB-Programmierung}

\subsection*{}
\begin{frame}{SIB-Programmierung}
	\begin{center}
		\includegraphics[width=\textheight]{Bilder/titel_sibs.png}
	\end{center}
\end{frame}


\begin{frame}{Vorüberlegungen...}
\begin{itemize}
	\item \textbf{3 Sorten von SIBs:} Google, SAP, GUI
	\pause
	\item \textbf{Google-SIBs:} Kontakt suchen und hinzufügen
	\item \textbf{SAP-SIB:} Kontakt suchen
	\pause
	\item \textbf{GUI-SIBs:}
		\begin{itemize}
			\item Eingabe von Kontakt-Attributen
			\item Auswahl aus einer Liste von Kontakten
		\end{itemize}

\end{itemize}
\end{frame}



\subsection*{SIBs im Detail}
\begin{frame}{Google + SAP: suche Kontakt}
\begin{itemize}
	\item \textbf{Input:} eine Instanz der Klasse \texttt{Contact}
		\begin{itemize}
			\item dient als Filter für die Suche
		\end{itemize}
	\item \textbf{Output:} Liste von \texttt{Contact}-Objekten

	\item \textbf{Branches:}
		\begin{itemize}
			\item \textbf{found:} mindestens ein Kontakt gefunden
			\item \textbf{not found:} keine Ergebnisse
			\item \textbf{error:} es wurde eine \texttt{Exception} geworfen
		\end{itemize}

\end{itemize}
\end{frame}


\begin{frame}{Google: Kontakt hinzufügen}
\begin{itemize}
	\item \textbf{Input:} eine Instanz der Klasse \texttt{Contact}

	\item \textbf{Output:} keiner

	\item \textbf{Branches:}
		\begin{itemize}
			\item \textbf{default:} Kontakt erfolgreich hinzugefügt
			\item \textbf{error:} es wurde eine \texttt{Exception} geworfen
		\end{itemize}

\end{itemize}
\end{frame}


\begin{frame}{GUI: Kontakt-Daten eingeben}
\begin{itemize}
	\item \textbf{Input:} eine Instanz der Klasse \texttt{Contact}
		\begin{itemize}
			\item Formular wird entsprechend befüllt
			\item zudem Parameter für Fenstertitel und Validierung der Eingabe
		\end{itemize}
	\pause
	\item \textbf{Output:} geänderte(!) Instanz der Klasse \texttt{Contact}
		\begin{itemize}
			\item wenn Button "`OK"' geklickt wurde
		\end{itemize}
	\pause
	\item \textbf{Branches:}
		\begin{itemize}
			\item \textbf{ok:} Eingabe bestätigt mit Button "`OK"'
			\item \textbf{cancel:} Eingabe abgebrochen mit Button "`CANCEL"'
			\item \textbf{error:} es wurde eine \texttt{Exception} geworfen, oder \texttt{UNKNOWN}
		\end{itemize}

\end{itemize}
\end{frame}


\begin{frame}{GUI: Kontakt-auswählen}
\begin{itemize}
	\item \textbf{Input:} Liste von \texttt{Contact}-Objekten
		\begin{itemize}
			\item zudem Parameter für Fenstertitel
		\end{itemize}
	\item \textbf{Output:} eine Instanz der Klasse \texttt{Contact}
		\begin{itemize}
			\item wenn Button "`OK"' geklickt wurde
		\end{itemize}

	\item \textbf{Branches:}
		\begin{itemize}
			\item \textbf{ok:} Eingabe bestätigt mit Button "`OK"'
			\item \textbf{cancel:} Eingabe abgebrochen mit Button "`CANCEL"'
			\item \textbf{error:} es wurde eine \texttt{Exception} geworfen, oder \texttt{UNKNOWN}
		\end{itemize}

\end{itemize}
\end{frame}


\subsection*{Probleme während der Entwicklung}
\begin{frame}[fragile]{Besonderheit der GUI-Programmierung}
\begin{itemize}
	\item \textbf{warten auf Eingabe:} wie \texttt{trace()}-Methode anhalten?
		\begin{itemize}
			\item Swing-Frame läuft in einem eigenem Thread!
		\end{itemize}
	\pause
	\item \textbf{Lösung hier:} mittels \texttt{synchronized}-Block in Frame und SIB
\end{itemize}

	\begin{block}{Thread-Synchronisation}
	\javalstset
	\begin{lstlisting}
	public String trace(ExecutionEnvironment env) {
		...
		synchronized (frame) {
			frame.wait();
			...
	\end{lstlisting}
	\end{block}

\begin{itemize}
	\item \textbf{Vorgehen:}
		\begin{itemize}
			\item \texttt{trace()} erstellt den Frame und wartet auf ein \texttt{notify()}
			\item \texttt{notify()} wird vom Action-Listener der Buttons aufgerufen
			\item im Anschluss kann \texttt{trace()} die Eingabe im Frame abfragen
		\end{itemize}
\end{itemize}
\end{frame}



\begin{frame}[fragile]{Eigener Code im jABC}
\begin{itemize}
	\item \textbf{Unterschiede in der Laufzeitumgebung:} 
	\begin{itemize}
			\item Code läuft ausserhalb von jABC...
			\item ABER: Ausführung des Graphen wirft \texttt{Exception}?
	\end{itemize}
	\pause
	\item \textbf{Lösung hier:} mittels "`\underline{bad} Practice"'
\end{itemize}

\begin{block}{Änderung von Systemeigenschaften}
\javalstset
\begin{lstlisting}
...
System.setProperty("javax.xml.parsers.SAXParser", ...);
System.setProperty("...parsers.SAXParserFactory", ...);
System.setProperty("oracle.xml.parser.v2.SAXParser...);
...
\end{lstlisting}
\end{block}

\begin{itemize}
	\item \textbf{Effekt:}
		\begin{itemize}
			\item konfiguriert aktiv laufende JVM
			\item auch nach Ausführung des Modells weiterhin wirksam
			\item sollte in realem Szenario vermieden werden!
		\end{itemize}
\end{itemize}
\end{frame}


\section{jABC-Modell}

\subsection*{}
\begin{frame}{Der Modell-Graph im jABC}
\begin{center}
\includegraphics[height=0.8\textheight]{Bilder/jabc_Model.png} 
\end{center}
\end{frame}






\begin{frame}
%%%%%%%%%%%%%%	Ende		%%%%%%%%%%%%%%
\begin{center}
Vielen Dank f\"ur die Aufmerksamkeit!
\end{center}
\end{frame}



\end{document}
