%%%%%%%%%%%%%%%%%%%%%%%%%%
% BITTE IN UTF-8 OEFFNEN %
%%%%%%%%%%%%%%%%%%%%%%%%%%
\documentclass[xcolor=dvipsnames, compress, 10pt]{beamer}

\usepackage[ngerman] {babel}
\usepackage[ansinew] {inputenc}

\usepackage{tikz}
\usepackage{graphics}
\usepackage{BeamerColor}
\usepackage{expdlist} 
\usepackage{eurosym}

\usetheme{Warsaw}
\usefonttheme{professionalfonts}

\usecolortheme[named=ForestGreen]{structure}

\usepackage{ownTheme}
\setbeamertemplate{footline}[infolines theme]
\setbeamertemplate{headline}[miniframes theme]
\setbeamercovered{transparent=8}

\AtBeginSubsection[]
{
\begin{frame}
\frametitle{}
\tableofcontents[currentsection]
\end{frame}
}

\begin{document}

\title[DLI Projekt: Kontakte]{DLI Projekt: Kontakte}
\subtitle[DLI]{Vorlesung Aktuelle Themen der Dienstleistungsinformatik}
%\date{\today} % gibt aktuelles Datum zurueck
\date{22. Januar 2013}
\author[M.\ Marzotko, T.\ Seeland, D. Wirkner]{Markus Marzotko, Thorben Seeland \& Dominic Wirkner\\ {\scriptsize Prof.\ Dr.\ Bernhard Steffen, Dipl.-Inf. Markus Doet}} 
\institute[TU Dortmund]{Technische Universit\"at Dortmund}

\nocite{*}

\frame{\titlepage}

\begin{frame}
%%%%%%%%%%%%%%	Gliederung		%%%%%%%%%%%%%%
\tableofcontents
\end{frame}			

\section{Einf\"uhrung}

\subsection*{Begriffskl\"arung}

\begin{frame}{Was ist Compliance?}
\pause
\begin{block}{Definition 1}
		\begin{quote}
			"`Die Binsenweisheit, dass Unternehmen Gesetze einhalten m\"ussen, hei{\ss}t nun
			Compliance."'
		\end{quote}
			Uwe H. Schneider, ZIP 2003, S. 645f
\end{block}
\pause
\begin{block}{Definition 2}
	\begin{quote}
		Bei "`Compliance"' geht es um die "`Erf\"ullung"', "`Entsprechung"' bzw.
		"`Konformit\"at"' mit staatlichen Gesetzen sowie mit Regeln und Spezifikationen,
		mit Grunds\"atzen (ethische und moralische) und Verfahren sowie mit Standards
		(z.B. ISO) und Konventionen, die klar definiert worden sind. Die Erf\"ullung der
		Compliance kann sowohl auf Zwang (z.B. durch Gesetze) als auch auf
		Freiwilligkeit (z.B. Einhaltung von Standards) beruhen.
	\end{quote}
		laut Compliance-Magazin.de
\end{block}
\end{frame}

\subsection*{Typische Anwendungsfelder}

\begin{frame}{Typische Anwendungsfelder}
\pause
\begin{itemize}[<+->]
	\item Medizin
		\begin{itemize}[<+->]
			\item ordnungsgem\"a{\ss}es Verschreiben von Medikamenten
			\item Befolgen der Ratschl\"age von \"Arzten
			\item Physiologie: Dehnbarkeit von K\"orperstrukturen
		\end{itemize}
	\item Cross-Compliance
		\begin{itemize}[<+->]
			\item landwirtschaftlicher Kontext
			\item betrifft Verpflichtungen f\"ur den Erhalt von EU-Direktbeihilfen:
				\begin{itemize}[<+->]
					\item landwirtschaftlicher und \"okologischer Zustand der Fl\"achen
					\item Futtermittel und Lebensmittelsicherheit
					\item Tiergesundheit und Tierschutz
				\end{itemize}
		\end{itemize}
	\item IT-Compliance
	\begin{itemize}[<+->]
		\item "`unsere"' Compliance
		\item Bedeutungsgewinn durch Digitalisierung von Gesch\"aftsprozessen
		\item Gesetze und Standards als Reaktion auf Computerpannen etc.
	\end{itemize}
\end{itemize}
\end{frame}

\section{Rechtliche Grundlagen}

\begin{frame}{Enron, Worldcom und Neuer Markt}
\begin{itemize}
	\pause
	\item \textbf{Enron} (USA, Energiekonzern)
	\pause
	\item \textbf{Worldcom} (USA, Telekommunikationskonzern)
	\pause
	\item \textbf{Neuer Markt}
\end{itemize}
\end{frame}

\subsection*{Sarbanes-Oxley Act}

%SAP 
\subsection{

}



\begin{frame}
%%%%%%%%%%%%%%	Ende		%%%%%%%%%%%%%%
\begin{center}
Vielen Dank f\"ur die Aufmerksamkeit!
\end{center}
\end{frame}

\end{document}
